\documentclass[12pt]{article}
\bibliographystyle{plain}
\usepackage{stmaryrd}
\title{Representation of $\lambda$-calculus in $\pi$-calculus}
\author{Jatin Shah}
\date{\today}

\newcommand{\trans}[3]{\ensuremath{\displaystyle #1 \stackrel{#2}{\longrightarrow} #3}}
\newtheorem{thm}{Theorem}[section]
\begin{document}
\maketitle
\section{Introduction}
$\pi$-calculus was introduced by Robin Milner \cite{PiCalc} as a natural way of representing communicating system. Through its various constructs, $\pi$-calculus can represent concurrency of processes. The Twelf \cite{Twelf} system can be used for proving different properties of various deductive systems. In this paper, I have described the implementation of $\pi$-calculus in Twelf. I have also attempted to represent linear $\lambda$-calculus in $\pi$-calculus and proved some properties of the representation using Twelf.
\section{$\pi$-Calculus}
In $\pi$-calculus \cite{PiCalc} there are only two primitive entities: {\it names} and {\it processes}. Let ${\cal N}$ be a infinite set of names, ranged over by $x$ and $y$. The set of processes ${\cal P}$, ranged over by $P$ and $Q$, is defined by following abstract syntax:
\[ P ::= 0 \> | \> \overline{x}y.P \> | \> x(y).P \> | \> \tau.P \> | \> (\nu x)P \> | \> !P \> | \> (P1|P2) \> | \> P1+P2 \> |\> [x=y]P \> | \> [x \neq y] P\]
The {\it input prefix} operator $x(y)$ and {\it restriction} operator $(\nu x)$ bind occurences of $y$ in $x(y).P$ and $(\nu y)P$ respectively. The process $!P$ behaves essentially as $P|P|P|\dots$ and the process $[x \neq y]P$ behaves as $P$ if $x$ and $y$ are different names.
\par There are four actions in $\pi$-calculus, defined by the syntax \[ \alpha ::= \tau \> | \> x(z)\> | \> \overline{x}z \> | \> \overline{x}(z) \] Their intuitive meaning is the following:
\begin{enumerate}
\item {\it silent} action: $P \stackrel{\tau}{\longrightarrow} Q$ means that $P$ can reduce itself into $Q$ without interacting with other processes.
\item {\it free output} action: $P \stackrel{\overline{x}y}{\longrightarrow} Q$ means that $P$ can reduce itself to $Q$ emitting the name $y$ on the channel $x$.
\item {\it input} action: $P \stackrel{x(z)}{\longrightarrow} Q$ means that $P$ can recieve from the channel $x$ any name $w$ and then evolve into $Q\{w/z\}$.
\item {\it bound output} action: $P \stackrel{\overline{x}(z)}{\longrightarrow} Q$ means that $P$ can evolve into $Q$ emitting on the channel $x$ a name $z$, which is bound in $P$.
\end{enumerate}
\par The late operational semantics \cite{Honsell} of $\pi$-calculus is given in table \ref{tbl:oper}.
\begin{table}
\begin{center}
\begin{tabular}{|lr|}
\hline
$IN \frac{\displaystyle -}{\trans{x(z).P}{x(w)}{P\{w/z\}}} \> w \notin fn((\nu z)P)$
 & $OUT \frac{\displaystyle -}{\trans {\overline{x}y.P}{\overline{x}y} P}$ \\
$PAR_1 \frac{\trans{P}{\alpha}{P'}}{\trans{P|Q}{\alpha}{P'|Q}} \> bn(\alpha) \cap fn(Q) = \phi$ 
& $ SUM_1 \frac{\trans{P}{\alpha}{P'}}{\trans{P+Q}{\alpha}{P'}}$ \\
$PAR_2 \frac{\trans{Q}{\alpha}{Q'}}{\trans{P|Q}{\alpha}{P|Q'}} \> bn(\alpha) \cap fn(P) = \phi$ 
& $SUM_2 \frac{\trans{Q}{\alpha}{Q'}}{\trans{P+Q}{\alpha}{Q'}}$ \\
$COM_1 \frac{\trans{P}{x(z)}{P'} \quad \trans{Q}{\overline{x}y}{Q'}}{\trans{P|Q}{\tau}{P'\{y/z\}|Q'}}$ 
& $COM_2 \frac{\trans{P}{\overline{x}y}{P'} \quad \trans{Q}{x(z)}{Q'}}{\trans{P|Q}{\tau}{P'|Q'\{y/z\}}}$ \\
$RES \frac{\trans{P}{\alpha}{P'}}{\trans{(\nu y) P}{\alpha}{(\nu y) P'}} \> y \notin n(\alpha)$ 
& $REPL \frac{\trans{P|!P}{\alpha}{Q}}{\trans{!P}{\alpha}{Q}}$ \\
$MATCH \frac{\trans{P}{\alpha}{P'}}{\trans{[x=x]P}{\alpha}{P'}}$ 
& $MISMATCH \frac{\trans{P}{\alpha}{P'}}{\trans{[x \neq y]P}{\alpha}{P'}} \> x \neq y$ \\
$OPEN \frac{\trans{P}{\overline{x}y}{P'}}{\trans{(\nu y) P}{\overline{x}(w)}{P'\{w/y\}}} \> y \neq x, w \notin fn((\nu y) P')$ 
& $TAU \frac{-}{\trans{\tau.P}{\tau}{P}}$ \\
$CLOSE_1 \frac{\trans{P}{x(w)}{P'} \quad \trans{Q}{\overline{x}(w)}{Q'}}{\trans{P|Q}{\tau}{(\nu w)(P'|Q')}}$ 
& $CLOSE_2 \frac{\trans{P}{\overline{x}(w)}{P'} \quad \trans{Q}{x(w)}{Q'}}{\trans{P|Q}{\tau}{(\nu w)(P'|Q')}}$ \\
\hline
\end{tabular}
\end{center}
\caption{Late Operational Semantics of $\pi$-calculus}
\label{tbl:oper}
\end{table}
\section{$\lambda$-Calculus in $\pi$-Calculus}
In $\lambda$-calculus there only terms and variables. Let $\cal V$ denote the set of variables ranged over by $x$. The set of terms, $\Lambda$, ranged over by $M$ is defined by \[ M ::= x \> | \> (\lambda x M) \> | \> (MN) \] The lazy operational semantics of $\lambda$-calculus is defined by 
\[ \frac{-}{{\lambda x.M} \hookrightarrow {\lambda x.M}} \> ev\_lam\]
	\[ \frac{M_1 \hookrightarrow \lambda x.M_1' \quad [M_2/x]M_1' \hookrightarrow V}{M_1M_2 \hookrightarrow V}\> ev\_app\]
	\par In linear $\lambda$-calculus each bound variable occurs only once. Thus, all terms in linear $\lambda$-calculus are linear terms. The translation of linear $\lambda$-calculus into $\pi$-calculus is shown below.
	\begin{eqnarray*}
	\llbracket \lambda x.M \rrbracket & \stackrel{def}{=} & u(x)(v).\llbracket M \rrbracket v \\
	\llbracket x \rrbracket u & \stackrel{def}{=} & \overline{x}u \\
	\llbracket M N \rrbracket u & \stackrel{def}{=} & (v)(\llbracket M \rrbracket v | (x) \overline{v}xu.x(w). \llbracket N \rrbracket w) \\
	& & (x \> not \> free \> in \> N) \\
	\end{eqnarray*}
	In the agent $\llbracket M \rrbracket u$, $u$ can be considered as pointing to the argument sequence appropriate for a particular occurence of $M$. In other words, if $M$ eventually reduces to a $\lambda$-abstraction $\lambda x.M'$, then the corresponding derivative of $\llbracket M \rrbracket u$ will receive along the link $u$ two names: a pointer to $M$'s first argument, and a pointer to the rest of its argument sequence.
	\par The double guarding of $\llbracket N \rrbracket$ is the essence of lazy evaluation. The first guard, the prefix $\overline{v}xu$, will be activated only when $M$ has reduced to the form $\lambda xM'$ and is ready for its argument; the second guard $x(w)$ will be activated only when $M'$ calls $N$ via the name $x$; only then may the reduction of $N$ commence.
\section{$\pi$-Calculus in Twelf}
Appendix \ref{sec:oper} gives the operational semantics of $\pi$-calculus in Twelf. There three different types of entities: processes, agents and actions. Processes are denoted by {\tt proc} and agents are denoted by {\tt name}. Actions are divided into two types: free actions denoted by {\tt f\_act} and bound actions denoted by {\tt b\_act}.
\section{Soundness of representation}
	In this section, I shall state the soundness of the representation of $\lambda$-calculus in $\pi$-calculus. The theorem is proved using Twelf.
\begin{thm}
{\bf (Soundness)}For any closed expressions $e$, $v$, processes $P$, $Q$, deductions ${\cal R}:: e \leftrightarrow P$, ${\cal T}::P \hookrightarrow Q$ and ${\cal S} :: Q \Leftrightarrow v$, there exists a deduction ${\cal D} :: e \hookrightarrow v$.
\end{thm}
\section*{Appendix}
\appendix
\section{Operational Semantics of $\pi$-calculus}
\label{sec:oper}
\begin{verbatim}
% Encoding of Processes
proc : type.
name : type.

nil           : proc.
!             : proc -> proc.                         %prefix 11 !.
tau_pref      : proc -> proc.
|             : proc -> proc -> proc.                 %infix left 10  |.
+             : proc -> proc -> proc.                 %infix left 9 +.
nu            : (name -> proc) -> proc.
match         : name -> name -> proc -> proc.
mismatch      : name -> name -> proc -> proc.
in_pref       : name -> (name -> proc) -> proc.
out_pref      : name -> name -> proc -> proc.

% Encoding of Actions
f_act : type.		% Free Action
b_act : type.		% Bound Action

% Free Actions
tau : f_act.
out : name -> name -> f_act.

% Bound Actions
in : name -> b_act.
bOut : name -> b_act.


% Operational semantics of the Pi-calculus
f_trans : proc -> f_act -> proc -> type.
b_trans : proc -> b_act -> (name -> proc) -> type.    


% Free Actions
fTAU    : f_trans (tau_pref P) tau P.
fOUT    : f_trans (out_pref X Y P) (out X Y) P.
fSUM1   : f_trans P1 A P -> f_trans (P1 + P2) A P.
fSUM2   : f_trans P2 A P -> f_trans (P1 + P2) A P.
fPAR1   : f_trans P1 A P -> f_trans (P1 | P2) A (P | P2).
fPAR2   : f_trans P2 A P -> f_trans (P1 | P2) A (P1 | P).
fMATCH  : f_trans P A Q -> f_trans (match X X P) A Q.
fBANG   : f_trans (P | (! P)) A Q -> f_trans (! P) A Q.
COM1    : b_trans P1 (in X) Q1 -> f_trans P2 (out X Y) Q2 -> 
          f_trans (P1 | P2) tau ((Q1 Y) | Q2). 
COM2    : b_trans P2 (in X) Q2 -> f_trans P1 (out X Y) Q1 -> 
fRES    : ({y:name} f_trans (P y) A (P' y)) -> f_trans (nu P) A (nu P'). 
CLOSE1  : b_trans P1 (in X) Q1 -> b_trans P2 (bOut X) Q2 -> 
          f_trans (P1 | P2) tau (nu [Z:name] ((Q1 Z) | (Q2 Z))).
CLOSE2  : b_trans P2 (in X) Q2 -> b_trans P1 (bOut X) Q1 -> 
          f_trans (P1 | P2) tau (nu [Z:name] ((Q1 Z) | (Q2 Z))).
SIMPL1  : f_trans (P | nil) tau P.
SIMPL2  : f_trans (nil | P) tau P.

% Bound Actions
IN      : b_trans (in_pref X P) (in X) P.
bSUM1   : b_trans P1 A P -> b_trans (P1 + P2) A P.
bSUM2   : b_trans P2 A P -> b_trans (P1 + P2) A P.
bPAR1   : b_trans P1 A P -> b_trans (P1 | P2) A ([X:name] ((P X) | P2)).
bPAR2   : b_trans P2 A P -> b_trans (P1 | P2) A ([X:name] (P1 | (P X))).
bMATCH  : b_trans P A Q -> b_trans (match X X P) A Q.
bBANG   : b_trans (P | (! P)) A Q -> b_trans (! P) A Q.
bRES    : ({y:name} b_trans (P y) A (P' y)) -> 
          b_trans (nu P) A  ([Z:name] (nu (P' Z))).
OPEN    : ({y:name} f_trans (P1 y) (out X y) (P2 y)) -> 
          b_trans (nu P1) (bOut X) P2.
\end{verbatim}

\section{Representation of $\lambda$-calculus in $\pi$-calculus}
\label{sec:rep}
\begin{verbatim}
% Lambda-calculus

exp : type.
app : exp -> exp -> exp.
lam : (exp -> exp) -> exp.


% Operational semantics of Lambda calculus

eval     : exp -> exp -> type.
eval_lam : eval (lam E) (lam E).
eval_app : eval (app E1 E2) V <- eval E1 (lam E1') <- eval (E1' E2) V.

% Embed Lambda calculus into Pi-calculus

conv     : exp -> (name -> proc) -> type.
conv_lam : conv (lam E) ([u:name] in_pref u ([z:name] (in_pref u (P z))))
           <- ({x:exp}{y:name} conv x ([u:name] (out_pref y u nil))  
                                             -> conv (E x) (P y)).


conv_app : conv (app E1 E2) ([u:name] nu ([v:name] (P1 v | 
                nu ([x:name] (out_pref v x (out_pref v u (in_pref x P2)))))))
                <- conv E1 P1
                <- conv E2 P2.
\end{verbatim}

\bibliography{picalculus}
\end{document}
